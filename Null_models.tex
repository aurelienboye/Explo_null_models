\documentclass[]{article}
\usepackage{lmodern}
\usepackage{amssymb,amsmath}
\usepackage{ifxetex,ifluatex}
\usepackage{fixltx2e} % provides \textsubscript
\ifnum 0\ifxetex 1\fi\ifluatex 1\fi=0 % if pdftex
  \usepackage[T1]{fontenc}
  \usepackage[utf8]{inputenc}
\else % if luatex or xelatex
  \ifxetex
    \usepackage{mathspec}
  \else
    \usepackage{fontspec}
  \fi
  \defaultfontfeatures{Ligatures=TeX,Scale=MatchLowercase}
\fi
% use upquote if available, for straight quotes in verbatim environments
\IfFileExists{upquote.sty}{\usepackage{upquote}}{}
% use microtype if available
\IfFileExists{microtype.sty}{%
\usepackage{microtype}
\UseMicrotypeSet[protrusion]{basicmath} % disable protrusion for tt fonts
}{}
\usepackage[margin=1in]{geometry}
\usepackage{hyperref}
\PassOptionsToPackage{usenames,dvipsnames}{color} % color is loaded by hyperref
\hypersetup{unicode=true,
            pdftitle={Méthodes de randomisation pour générer des modèles nuls dans le cadre d'analyse de sur- et sous dispersion des traits},
            pdfauthor={A. Boyé},
            colorlinks=true,
            linkcolor=Maroon,
            citecolor=red,
            urlcolor=blue,
            breaklinks=true}
\urlstyle{same}  % don't use monospace font for urls
\usepackage{color}
\usepackage{fancyvrb}
\newcommand{\VerbBar}{|}
\newcommand{\VERB}{\Verb[commandchars=\\\{\}]}
\DefineVerbatimEnvironment{Highlighting}{Verbatim}{commandchars=\\\{\}}
% Add ',fontsize=\small' for more characters per line
\usepackage{framed}
\definecolor{shadecolor}{RGB}{48,48,48}
\newenvironment{Shaded}{\begin{snugshade}}{\end{snugshade}}
\newcommand{\KeywordTok}[1]{\textcolor[rgb]{0.94,0.87,0.69}{#1}}
\newcommand{\DataTypeTok}[1]{\textcolor[rgb]{0.87,0.87,0.75}{#1}}
\newcommand{\DecValTok}[1]{\textcolor[rgb]{0.86,0.86,0.80}{#1}}
\newcommand{\BaseNTok}[1]{\textcolor[rgb]{0.86,0.64,0.64}{#1}}
\newcommand{\FloatTok}[1]{\textcolor[rgb]{0.75,0.75,0.82}{#1}}
\newcommand{\ConstantTok}[1]{\textcolor[rgb]{0.86,0.64,0.64}{\textbf{#1}}}
\newcommand{\CharTok}[1]{\textcolor[rgb]{0.86,0.64,0.64}{#1}}
\newcommand{\SpecialCharTok}[1]{\textcolor[rgb]{0.86,0.64,0.64}{#1}}
\newcommand{\StringTok}[1]{\textcolor[rgb]{0.80,0.58,0.58}{#1}}
\newcommand{\VerbatimStringTok}[1]{\textcolor[rgb]{0.80,0.58,0.58}{#1}}
\newcommand{\SpecialStringTok}[1]{\textcolor[rgb]{0.80,0.58,0.58}{#1}}
\newcommand{\ImportTok}[1]{\textcolor[rgb]{0.80,0.80,0.80}{#1}}
\newcommand{\CommentTok}[1]{\textcolor[rgb]{0.50,0.62,0.50}{#1}}
\newcommand{\DocumentationTok}[1]{\textcolor[rgb]{0.50,0.62,0.50}{#1}}
\newcommand{\AnnotationTok}[1]{\textcolor[rgb]{0.50,0.62,0.50}{\textbf{#1}}}
\newcommand{\CommentVarTok}[1]{\textcolor[rgb]{0.50,0.62,0.50}{\textbf{#1}}}
\newcommand{\OtherTok}[1]{\textcolor[rgb]{0.94,0.94,0.56}{#1}}
\newcommand{\FunctionTok}[1]{\textcolor[rgb]{0.94,0.94,0.56}{#1}}
\newcommand{\VariableTok}[1]{\textcolor[rgb]{0.80,0.80,0.80}{#1}}
\newcommand{\ControlFlowTok}[1]{\textcolor[rgb]{0.94,0.87,0.69}{#1}}
\newcommand{\OperatorTok}[1]{\textcolor[rgb]{0.94,0.94,0.82}{#1}}
\newcommand{\BuiltInTok}[1]{\textcolor[rgb]{0.80,0.80,0.80}{#1}}
\newcommand{\ExtensionTok}[1]{\textcolor[rgb]{0.80,0.80,0.80}{#1}}
\newcommand{\PreprocessorTok}[1]{\textcolor[rgb]{1.00,0.81,0.69}{\textbf{#1}}}
\newcommand{\AttributeTok}[1]{\textcolor[rgb]{0.80,0.80,0.80}{#1}}
\newcommand{\RegionMarkerTok}[1]{\textcolor[rgb]{0.80,0.80,0.80}{#1}}
\newcommand{\InformationTok}[1]{\textcolor[rgb]{0.50,0.62,0.50}{\textbf{#1}}}
\newcommand{\WarningTok}[1]{\textcolor[rgb]{0.50,0.62,0.50}{\textbf{#1}}}
\newcommand{\AlertTok}[1]{\textcolor[rgb]{1.00,0.81,0.69}{#1}}
\newcommand{\ErrorTok}[1]{\textcolor[rgb]{0.76,0.75,0.62}{#1}}
\newcommand{\NormalTok}[1]{\textcolor[rgb]{0.80,0.80,0.80}{#1}}
\usepackage{graphicx,grffile}
\makeatletter
\def\maxwidth{\ifdim\Gin@nat@width>\linewidth\linewidth\else\Gin@nat@width\fi}
\def\maxheight{\ifdim\Gin@nat@height>\textheight\textheight\else\Gin@nat@height\fi}
\makeatother
% Scale images if necessary, so that they will not overflow the page
% margins by default, and it is still possible to overwrite the defaults
% using explicit options in \includegraphics[width, height, ...]{}
\setkeys{Gin}{width=\maxwidth,height=\maxheight,keepaspectratio}
\IfFileExists{parskip.sty}{%
\usepackage{parskip}
}{% else
\setlength{\parindent}{0pt}
\setlength{\parskip}{6pt plus 2pt minus 1pt}
}
\setlength{\emergencystretch}{3em}  % prevent overfull lines
\providecommand{\tightlist}{%
  \setlength{\itemsep}{0pt}\setlength{\parskip}{0pt}}
\setcounter{secnumdepth}{0}
% Redefines (sub)paragraphs to behave more like sections
\ifx\paragraph\undefined\else
\let\oldparagraph\paragraph
\renewcommand{\paragraph}[1]{\oldparagraph{#1}\mbox{}}
\fi
\ifx\subparagraph\undefined\else
\let\oldsubparagraph\subparagraph
\renewcommand{\subparagraph}[1]{\oldsubparagraph{#1}\mbox{}}
\fi

%%% Use protect on footnotes to avoid problems with footnotes in titles
\let\rmarkdownfootnote\footnote%
\def\footnote{\protect\rmarkdownfootnote}

%%% Change title format to be more compact
\usepackage{titling}

% Create subtitle command for use in maketitle
\newcommand{\subtitle}[1]{
  \posttitle{
    \begin{center}\large#1\end{center}
    }
}

\setlength{\droptitle}{-2em}
  \title{Méthodes de randomisation pour générer des modèles nuls dans le cadre
d'analyse de sur- et sous dispersion des traits}
  \pretitle{\vspace{\droptitle}\centering\huge}
  \posttitle{\par}
  \author{A. Boyé}
  \preauthor{\centering\large\emph}
  \postauthor{\par}
  \predate{\centering\large\emph}
  \postdate{\par}
  \date{24 avril, 2018}

\usepackage[francais]{babel}
\usepackage[utf8]{inputenc}
\usepackage[T1]{fontenc}
\usepackage{fontspec}

\begin{document}
\maketitle

{
\hypersetup{linkcolor=black}
\setcounter{tocdepth}{2}
\tableofcontents
}
\begin{center}\rule{0.5\linewidth}{\linethickness}\end{center}

\section{\texorpdfstring{ \textbf{Idée principale}
}{ Idée principale }}\label{idee-principale}

Randomizer nos communautés avec certaines contraintes pour voir l'écart
entre FD observée et FD simulée \(\Rightarrow\) écart entre distribution
observée des traits et distribution attendu sous hypothèse nulle de
distribution random.

\begin{itemize}
\tightlist
\item
  \textbf{Overdispersion} : serait témoin de l'effet des intéractions
  biotiques qui a tendance à favoriser la distinction des niches entre
  espèces ainsi qu'une plus grande régularité dans l'espace des traits
  (\emph{limiting similarity})

  \begin{itemize}
  \tightlist
  \item
    Voir Mouillot \emph{et al.}
    (\protect\hyperlink{ref-mouillot2013indic}{2013})
  \item
    Voir discussion D'andrea \& Ostling
    (\protect\hyperlink{ref-dandrea_ostling2016}{2016}) sur les limites
    de cette interprétation et sur les patrons à attendre
  \end{itemize}
\item
  \textbf{Underdispersion} : témoin de l'effet de filtres
  environnementaux

  \begin{itemize}
  \tightlist
  \item
    si c'est le cas on peut pour aller plus loin et comprendre les
    facteurs à l'origine de cette underdispersion faire :

    \begin{itemize}
    \tightlist
    \item
      RLQ
    \item
      quatrième coin (lien individuel entre trait et envir qui colle
      assez bien à la démarche individuelle de cette analyse)
    \item
      ou varpart sur matrice abondance de trait avec \(envir + MEM\) :
      voir Zhang \emph{et al.} (\protect\hyperlink{ref-zhang2018}{2018})
      ou Astor \emph{et al.}
      (\protect\hyperlink{ref-astor2014trait_dispersion}{2014}) pour
      CWM-RDA; dont avantages sont décrits par Kleyer \emph{et al.}
      (\protect\hyperlink{ref-kleyer2012methods}{2012})

      \begin{itemize}
      \tightlist
      \item
        ou developper partition variance avec RLQ à partir de RLQ
        partielle (Wesuls \emph{et al.}
        \protect\hyperlink{ref-wesuls2012partialRLQ}{2012})
      \end{itemize}
    \end{itemize}
  \end{itemize}
\end{itemize}

\begin{quote}
\textbf{Attention} : cela est différent de modèle neutre qui décrit les
espèces comme équivalente et explique leur pattrons par les processus de
dispersion, l'état démographique des populations et la distribution
spatiale des espèces mais qui nécessite un paramétrage impossible à
faire dans notre cas. Utiliser des modèles nulles - randomization stat -
ou des modèles neutres amènent souvent des résultats différents, et il y
a des partisans de chacune de ces méthodes (voir D'andrea \& Ostling
\protect\hyperlink{ref-dandrea_ostling2016}{2016})
\end{quote}

\begin{center}\rule{0.5\linewidth}{\linethickness}\end{center}

\section{\texorpdfstring{ \textbf{Méthode} }{ Méthode }}\label{methode}

\subsection{\texorpdfstring{\textbf{Calcul de l'écart avec les
communautées
simulées}}{Calcul de l'écart avec les communautées simulées}}\label{calcul-de-lecart-avec-les-communautees-simulees}

L'écart entre les observations et les modèles nuls peut notamment être
calculé par Standard size effect (Gotelli \& McCabe
\protect\hyperlink{ref-gotelli_mccabe2002}{2002}) :

\(SES = \frac{FD_{observed} - \mu_{null}}{\sigma_{null}}\)

\begin{center}\rule{0.5\linewidth}{\linethickness}\end{center}

\subsection{\texorpdfstring{\textbf{Indice fonctionnel à
utiliser}}{Indice fonctionnel à utiliser}}\label{indice-fonctionnel-a-utiliser}

Le plus souvent cela est fait sur l'indice \(RaoQ\) (\emph{e.g.} Astor
\emph{et al.} \protect\hyperlink{ref-astor2014trait_dispersion}{2014};
Zhang \emph{et al.} \protect\hyperlink{ref-zhang2018}{2018}) mais cela a
aussi été fait avec le \(FRic\) et la \(FDis\) (Laliberté \emph{et al.}
\protect\hyperlink{ref-laliberte2013}{2013}; Mason \emph{et al.}
\protect\hyperlink{ref-mason2013}{2013}). Potentiellement cela peut etre
fait avec l'indice de \(distinctiveness\) de Violle \emph{et al.}
(\protect\hyperlink{ref-violle2017functional_rarity}{2017}) qui se
trouve dans le package \texttt{funrar} de Grenié \emph{et al.}
(\protect\hyperlink{ref-grenie2017funrar}{2017})

\begin{quote}
\(Functional~distinctiveness\) : mean functional distance of species
\(i\) to the \(N\) other species. Can be weighted by species relative
abundance because a species i even more distinct if it does not share
traits with the most abundant species within the community (Violle
\emph{et al.}
\protect\hyperlink{ref-violle2017functional_rarity}{2017}). This measure
can then be averaged at the scale of the community (see Grenié \emph{et
al.} \protect\hyperlink{ref-grenie2017funrar}{2017})
\end{quote}

\subsubsection{Bilan}\label{bilan}

\begin{itemize}
\tightlist
\item
  Favoriser \(RaoQ\) ou \(FDis\) par rapport à \(FRic\)
  (podani2009convex\_hulls; in Botta-Dukát \& Czúcz
  \protect\hyperlink{ref-botta_dukat2016}{2016})
\item
  Favoriser \(FDis\) car moins sensibles aux espèces aux traits extrêmes
  ? (Laliberté \& Legendre
  \protect\hyperlink{ref-laliberte2010FDis}{2010}; Laliberté \emph{et
  al.} \protect\hyperlink{ref-laliberte2013}{2013})
\item
  \(RaoQ\) pourrait permetre d'évaluer en même temps \(SES_{FRed}\) sur
  la redondance en simulant \(1 - \frac{RaoQ}{SimD}\)
\end{itemize}

\begin{center}\rule{0.5\linewidth}{\linethickness}\end{center}

\subsection{\texorpdfstring{\textbf{Analyse \emph{single-trait} versus
\emph{mutli-trait}}}{Analyse single-trait versus mutli-trait}}\label{analyse-single-trait-versus-mutli-trait}

Le calcule de l'indice de diversité sur les données observées ou
simulées peut se faire sur chaque trait séparément (\emph{e.g.} Astor
\emph{et al.} \protect\hyperlink{ref-astor2014trait_dispersion}{2014};
ou Zhang \emph{et al.} \protect\hyperlink{ref-zhang2018}{2018}) ou sur
l'ensemble des traits simultanément (voir Perronne \emph{et al.}
\protect\hyperlink{ref-perronne2017}{2017} pour discussion du sujet)

\begin{quote}
Assembly processes may act contrastingly on different traits (Spasojevic
\& Suding \protect\hyperlink{ref-spasojevic2012}{2012}). Indeed,
Spasojevic \& Suding (\protect\hyperlink{ref-spasojevic2012}{2012}) show
that multiple assembly processes (abiotic filtering, above‐ground
competition, and below‐ground competition) operated simultaneously to
structure plant communities and that they were therefore obscured by a
single multivariate trait index and only evident by analysing functional
diversity patterns of individual traits.
\end{quote}

\begin{quote}
Perronne \emph{et al.} (\protect\hyperlink{ref-perronne2017}{2017}) :
although metrics have been proposed for both single-trait and
multi-trait approaches, results from multivariate approaches may be more
difficult to understand, especially when several processes can lead to
fitness equalization via trade-offs among traits in multi-dimensional
niches (Clark et al., 2007). In addition, the use of a multi-trait
approach reveals several practi- cal restrictions (Lepsˇ et al., 2006)
leading to a lower detection of the signature of the underlying
mechanisms (Aiba et al., 2013). For instance, correlations between
traits may require removing some redundant traits, and using different
types of variables needs appro- priate standardization or transformation
(Lepsˇ et al., 2006; Petchey and Gaston, 2006; Bernhardt-Römermann et
al., 2008).
\end{quote}

\subsubsection{Bilan}\label{bilan-1}

\begin{itemize}
\tightlist
\item
  le faire en \emph{single\_trait} pour \(RaoQ\) et en
  \emph{multi-trait} pour \(FRed\), potentiellement en séparant trait
  effet et réponse ?
\end{itemize}

\begin{center}\rule{0.5\linewidth}{\linethickness}\end{center}

\subsection{\texorpdfstring{\textbf{Pool d'espèce sur lequel randomiser
?}}{Pool d'espèce sur lequel randomiser ?}}\label{pool-despece-sur-lequel-randomiser}

\begin{itemize}
\item
  Randomiser au sein de chaque habitat avec le pool d'espèce capable de
  vivre dans cet habitat \textbf{ou} en intertidal et subtidal
  séparément mais en randomisant entre les habitats biogéniques et nus
  au sein de chaque frange tidale ?
\item
  Le faire aussi pour chaque année séparemment ou randomiser toutes
  années confondus en considérant qu'il y a un pool d'espèces régional
  et que certaines années on en loupe ?
\end{itemize}

Selon Perronne \emph{et al.} (\protect\hyperlink{ref-perronne2017}{2017}
: p.38) \emph{``a reference species pool includes all species that may
potentially disperse to a specific site''}. Certaines études prennent
même en compte dans leur randomisation des espèces issues d'inventaires
régionaux mais pas échantillonnées dans leur analyse.

The study of environmental filtering should be done at broad spatial
scale, encompassing contrasted environments and with randomization
technique that randomize species accross the different environmental
conditions (Götzenberger \emph{et al.}
\protect\hyperlink{ref-gotzenberger2016randomizations}{2016}; Perronne
\emph{et al.} \protect\hyperlink{ref-perronne2017}{2017})

\subsubsection{Bilan}\label{bilan-2}

\begin{itemize}
\tightlist
\item
  Prendre donc dans la randomisation toutes les années confondus pour
  représenter ainsi le pool d'espèces trouvé dans la région ansi que
  pour chaque niveau tidal les deux habitats ensemble dans la
  randomisation \emph{i.e.} randomiser entre HI et IM ensemble et entre
  SM et MA ensemble car les espèces trouvées dans chaque habitat aurait
  le potentiel de se trouver dans l'autre si ce n'était du fait de
  filtre environnementaux ou de competition (\emph{i.e.} les processus
  qu'on test)
\end{itemize}

\textbf{Attention} : Cette méthode permet surtout de tester l'effet des
filtres environnementaux, pour tester l'effet de la competition
\emph{i.e.} \textbf{limiting similarity}, il faut randomiser les espèces
avec un pool d'espèces qui se limite aux espèces pouvant persister dans
l'environnement. Cela peut etre fait en randomisant au sein unité
géographique définies par exemple (Perronne \emph{et al.}
\protect\hyperlink{ref-perronne2017}{2017} : p.38) \(\Rightarrow\)
\textbf{randomiser au sein des sites}

\begin{quote}
Chalmandrier \emph{et al.}
(\protect\hyperlink{ref-chalmandrier2013}{2013}) : Large scales thereby
reinforce the detection of the effect of environ- mental gradients,
while a small spatial scale is better sui- ted to detect competition
(Thuiller et al. 2010; Mouquet et al. 2012).
\end{quote}

\begin{quote}
Limiting similarity and competitive dominance are harder to detect and
should be assessed in homogenous environments at fine scale (Botta-Dukát
\& Czúcz \protect\hyperlink{ref-botta_dukat2016}{2016}; Götzenberger
\emph{et al.}
\protect\hyperlink{ref-gotzenberger2016randomizations}{2016}; Perronne
\emph{et al.} \protect\hyperlink{ref-perronne2017}{2017}). Species
should not be randomize between sites from different environmental
conditions (Götzenberger \emph{et al.}
\protect\hyperlink{ref-gotzenberger2016randomizations}{2016})
\end{quote}

\begin{quote}
Perronne \emph{et al.} (\protect\hyperlink{ref-perronne2017}{2017}) :
Alternatively, when investigating the signatures of more local
ecological processes in a particular environmental context,
e.g.~studying trait distribution patterns associated with fine-scale
disturbances and interactions, the species pool needs to be limited to a
subset of species able to persist in these particular environmental
condi- tions, i.e.~presenting specific functional traits and properties,
such as an appropriate phenology or some tolerance or avoidance abil-
ities regarding a key abiotic factor (de Bello et al., 2012; Lessard et
al., 2012). The pool can then be restricted to the species observed in
each sampling unit and ignoring the species from the other sites that
constitute the matrix (Bernard-Verdier et al., 2012; Table 4; see
Supplementary Appendix B and C for details), potentially includ- ing
species from neighboring sources (Lessard et al., 2012; Lessard et al.,
2016), or species occurring at regional scale presenting val- ues within
the trait range of each particular site, i.e.~assuming that species with
intermediate trait values could be adapted to local con- ditions of the
site considered (e.g.~Gotelli and Ellison, 2002; Table 4; Fig. 3A). The
pool can also be defined based on operational sub- matrices consisting
of environmentally uniform groups of sites, such as types of vegetation
(Ding et al., 2012; Pipenbaher et al., 2013) or locations (Richardson et
al., 2012), also termed geograph- ical units (Cornell and Harrison,
2014). These options specifically addresses the influence of local
niche-based mechanisms while controlling the influence of environmental
filtering playing at a larger habitat scale
\end{quote}

\begin{center}\rule{0.5\linewidth}{\linethickness}\end{center}

\subsection{\texorpdfstring{\textbf{Quelle type de randomisation pour le
modèle
nul}}{Quelle type de randomisation pour le modèle nul}}\label{quelle-type-de-randomisation-pour-le-modele-nul}

\begin{itemize}
\tightlist
\item
  Maintenir richesse fixe
\end{itemize}

\begin{quote}
``\emph{For instance, modifying species richness at a site, i.e.~row
marginal sums, will impact the values of most functional metrics
currently available (Mouchet et al., 2010), leading to interpretations
which are not only influenced by the mechanism studied, but also by the
differences in species richness between observed and virtual communities
caused by a methodological choice. In this case, the explicit (and more
often implicit) principle of not holding constant the species richness
of each community can be seen as a computational bias}'' (Perronne
\emph{et al.} \protect\hyperlink{ref-perronne2017}{2017})
\end{quote}

\begin{itemize}
\tightlist
\item
  Best model for testing environmental filtering : C4, C5, T3 and T4
  (Götzenberger \emph{et al.}
  \protect\hyperlink{ref-gotzenberger2016randomizations}{2016})
\end{itemize}

\begin{quote}
\textbf{C4} : randomize sample abundances within each species across all
samples (`frequency' randomization \emph{sensu} \texttt{Picante})

\begin{itemize}
\tightlist
\item
  Richness of samples \textbf{is not} fixed
\item
  Total abundances per sample \textbf{is not} fixed
\item
  Matrix position \textbf{is not} fixed (``\emph{when matrix position is
  fixed in a randomization, the species that occur in a local community
  do not change, only their abundances do}'')
\item
  Number of samples inhabited by each species is fixed (``\emph{fixed
  frequency}'')
\item
  Total species abundance is fixed (``\emph{abondance totale de chaque
  espèces sur l'ensemble des échantillons}'')
\end{itemize}

\textbf{C5} : randomize species' presence with their abundances

\begin{itemize}
\tightlist
\item
  Richness of samples is fixed
\item
  Total abundances per sample \textbf{is not} fixed
\item
  Matrix position \textbf{is not} fixed (``\emph{when matrix position is
  fixed in a randomization, the species that occur in a local community
  do not change, only their abundances do}'')
\item
  Number of samples inhabited by each species is fixed (``\emph{fixed
  frequency}'')
\item
  Total species abundance is fixed (``\emph{abondance totale de chaque
  espèces sur l'ensemble des échantillons}'')
\end{itemize}

Algorithme Trialswap de la fonction\texttt{randomizeMatrix} de
\texttt{Picante}\(\Rightarrow\) \textbf{fixed richness} \(\oplus\)
\textbf{fixed colSums : total abundances of species} \(\oplus\)
\textbf{fixed number of occurence per species}
\end{quote}

\begin{itemize}
\tightlist
\item
  Best model for testing limiting similarity : C2, T1 and T2
  (\textbf{ATTENTION} : ``In this case, one should never use
  randomizations that swap abundances or occurrences betwwen sites from
  different environments, or trait values between species that occur in
  these different environments'', Götzenberger \emph{et al.}
  \protect\hyperlink{ref-gotzenberger2016randomizations}{2016})
\end{itemize}

\begin{quote}
\textbf{A DECRIRE}
\end{quote}

\subsection{\texorpdfstring{\texttt{R}
functions}{R functions}}\label{r-functions}

\begin{itemize}
\tightlist
\item
  \texttt{permatswap} / \texttt{commsimmulator}
\end{itemize}

In \texttt{permatswap}, a special swap algorithm (`swapcount') is
implemented that results in permuted matrices with fixed marginals and
matrix fill at the same time \(\Rightarrow\) fixed total abundances per
samples \(\oplus\) fixed total abundances per species \(\oplus\) fixed
porportion of empty cells. \textbf{WARNING}: according to simulations,
this algorithm seems to be biased and non random, thus its use should be
avoided!

The algorithm ``abuswap'' produces two kinds of null models (based on
fixedmar=``columns'' or fixedmar=``rows'') as described in Hardy (2008;
randomization scheme 2x and 3x, respectively). These preserve column and
row occurrences, and column or row sums at the same time. Note that
similar constraints can be achieved by the non sequential ``swsh''
algorithm with fixedmar argument set to ``columns'' or ``rows'',
respectively. \(\Rightarrow\) \textbf{pour faire modèle C5 de
Götzenberger \emph{et al.}
(\protect\hyperlink{ref-gotzenberger2016randomizations}{2016}) ?}

\textbf{Exemples d'utilisation}

Bello (\protect\hyperlink{ref-de_Bello_2011}{2011}) : ``The results
presented refer to the quasiswap method of Miklos \& Podani (2004), with
499 randomizations, which satisfies the require- ments for
equidistribution of species. The commsimulator function was used to
generate null communities (vegan library; R Development Core Team,
2009). The quasiswap method ran- domizes species composition while
keeping the number of species per plot in the randomized data fixed (see
de Bello et al., 2009, for details).''

\begin{quote}
Maintain row and column totals selon l'aide de la fonction : sur
présence absence cela préserve la richesse des échantillons mais les
données d'abondance, cela devrait préserver les abondances totales des
échantillons. Comportement de cette fonction à vérifier
\end{quote}

Laliberté \emph{et al.} (\protect\hyperlink{ref-laliberte2013}{2013}) :
``The randomization was done using the `commsimulator' function (`r1′
algorithm) in the vegan R package (R Foundation for Statistical Com-
puting, Vienna, AT).''

\begin{quote}
``r1'': non-sequential algorithm for binary matrices that preserves the
site (row) frequencies, but uses column marginal frequencies as
probabilities of selecting species.
\end{quote}

\begin{itemize}
\tightlist
\item
  \texttt{randomizeMatrix} du package \texttt{Picante} (voir aussi
  \texttt{phylostruct} )
\end{itemize}

Currently implemented null models (arguments to null.model):
\textgreater{} \emph{frequency} : Randomize community data matrix
abundances within species (maintains species occurence frequency)
\textgreater{} \emph{richness} : Randomize community data matrix
abundances within samples (maintains sample species richness)
\textgreater{} \emph{independentswap} : Randomize community data matrix
with the independent swap algorithm (Gotelli 2000) maintaining species
occurrence frequency and sample species richness \textgreater{}
\emph{trialswap} : Randomize community data matrix with the trial-swap
algorithm (Miklos \& Podani 2004) maintaining species occurrence
frequency and sample species richness

Le comportement de ces différents algorithme est testé dans la partie
\protect\hyperlink{test}{Test}

\subsection{Test de significativité}\label{test-de-significativite}

Zhang \emph{et al.} (\protect\hyperlink{ref-zhang2018}{2018}) utilisent
un one-sample Wilcoxon signed-rank test. Ils ne précisent par-contre pas
si le test est fait en uni- ou bi-latéral. Dans Götzenberger \emph{et
al.} (\protect\hyperlink{ref-gotzenberger2016randomizations}{2016}) il
le font (avec un test different) en uni-latéral (\emph{i.e.} one-sided)
bien qu'ils test en unilatéral des deux cotés avec la même distribution.
Cette méthode semble donc inapropriée car elle accroit le risque
\(\alpha\) pour ce qui parait être un test bilatérale.

\begin{quote}
\texttt{wilcox.test(x,\ y\ =\ NULL,alternative\ =\ c("two.sided",\ "less",\ "greater"),mu\ =\ 0)}
\end{quote}

\(\square\) Vérifier comment fonctionne se test

\(\square\) Analyser comment le test de significativité se fait dans les
autres papiers

\begin{center}\rule{0.5\linewidth}{\linethickness}\end{center}

\section{\texorpdfstring{ \textbf{Autres notes importantes}
}{ Autres notes importantes }}\label{autres-notes-importantes}

When trait en related to both filtering and competition, it is virtually
impossible to tease environmental filtering or competitive dominance
apart (Götzenberger \emph{et al.}
\protect\hyperlink{ref-gotzenberger2016randomizations}{2016})
\(\Rightarrow\) \textbf{HI ?}

\begin{quote}
Utiliser set.seed : ``Initially, there is no seed; a new one is created
from the current time and the process ID when one is required. Hence
different sessions will give different simulation results, by default.
However, the seed might be restored from a previous session if a
previously saved workspace is restored.'', this is why you would want to
call set.seed() with same integer values the next time you want a same
sequence of random sequence."
\end{quote}

\begin{center}\rule{0.5\linewidth}{\linethickness}\end{center}

\section{\texorpdfstring{ \textbf{Extraits méthodes dans la littérature}
}{ Extraits méthodes dans la littérature }}\label{exemples}

\begin{itemize}
\tightlist
\item
  Zhang \emph{et al.} (\protect\hyperlink{ref-zhang2018}{2018})
\end{itemize}

\begin{quote}
Reshuffling the species × quadrat matrices in each spatial scale was
done with three constraints simultaneously, following the method of
Zhang et al. (2015), i.e., keeping: i) the same number of species
(species richness) per plot in the simulated and observed data; ii) the
same number of total species occurrences per region (i.e., number of
plots where the species occur in each group of the five spatial scales);
and iii) the total abundance of species in a region constant (i.e., the
sum of the number of quadrats occupied in all plots). We implemented
this using the function ``randomizeMatrix'' in the ``picante'' package
in R (Kembel et al. 2010). We then compared the observed FD to the FD
simulated in 1000 randomly assembled communities. The Standard Effect
Size index (SES) following Gotelli \& McCabe (2002) was used as a
measure for FD patterns.{[}\ldots{}{]} We used the Wilcoxon signed-rank
tests to examine whether SES is significantly more than, less than or
approximately equal to zero, which indicates the prevalence of
significant trait divergence, trait convergence, and random
distribution, respectively.

\begin{itemize}
\tightlist
\item
  Keep site richness \(+\) same total species occurences \(+\) same
  species total abundance \(\Rightarrow\) model C5 of Götzenberger
  \emph{et al.}
  (\protect\hyperlink{ref-gotzenberger2016randomizations}{2016}).

  \begin{itemize}
  \tightlist
  \item
    This model appears the best to evaluate the signature of
    environmental filtering according to Götzenberger \emph{et al.}
    (\protect\hyperlink{ref-gotzenberger2016randomizations}{2016}) with
    the model C4 but is potentially more appropriate than C4 as it keeps
    richness constant contrary to model C4, which is a desirable
    property for the randomization (see appendix 1 of Perronne \emph{et
    al.} \protect\hyperlink{ref-perronne2017}{2017})
  \end{itemize}
\item
  Fait avec la fonction \texttt{randomizeMatrix} du package
  \texttt{Picante}. Dans Zhang \emph{et al.}
  (\protect\hyperlink{ref-zhang2015}{2015}), ils ont utilisé la fonction
  \texttt{commsimulator} de \texttt{vegan} with the ``trialswap''
  method. Existe aussi dans \texttt{randomizeMatrix} mais selon l'aide
  préserve seulement species occurrence et sample richness. J'ai testé
  cette méthode (voir \protect\hyperlink{test}{Test}), et elle semble
  bien aussi préservé l'abondance totale de chaque espèce.
\end{itemize}
\end{quote}

\begin{itemize}
\tightlist
\item
  Laliberté \emph{et al.} (\protect\hyperlink{ref-laliberte2013}{2013})
\end{itemize}

\begin{quote}
\begin{itemize}
\tightlist
\item
  \(FRic\)

  \begin{itemize}
  \tightlist
  \item
    We used a null model approach to test whether the observed
    functional richness in each plot was lower or higher than expected
    if metacommunity assembly was not influenced by species traits
    (i.e.~leaf {[}N{]} and plant height). To do so, we created 999
    random species compo- sition matrices where species richness per
    plot (i.e.~row) was preserved, and where column marginal frequencies
    were used as probabilities. The randomization was done using the
    \texttt{commsimulator} function (`r1′ algorithm) in the vegan R
    package (R Foundation for Statistical Com- puting, Vienna, AT). We
    then computed standardized deviations from the null expectation
    (FRicdev) for each plot
  \end{itemize}
\item
  \(FDis\)

  \begin{itemize}
  \tightlist
  \item
    \(FDis\) is related to Rao's quadratic entropy (Botta-Dukat 2005;
    Laliberte \& Legendre 2010), which is the multivariate analogue of
    the weighted variance, but \textbf{\(FDis\) is by construction less
    sensitive than \(Rao’s Q\) to species with extreme trait values.}
  \item
    To test for trait divergence, we used a different null model than
    that described above for functional richness. First, and most
    importantly, we restricted the randomization between 1 m 9 1-m
    quadrats within each 8 m 9 50-m plot, in order to detect local-scale
    assembly processes independent of any filtering effect on species
    pools (Mason et al. 2011; de Bello et al. 2012). To do so, we kept
    species abundance matrices for each plot (i.e.~20 quadrats as rows)
    constant but randomized rows within the species 9 trait matrix
    (Stubbs \& Wilson 2004), using 999 randomizations.
  \end{itemize}
\end{itemize}
\end{quote}

\begin{itemize}
\tightlist
\item
  Astor \emph{et al.}
  (\protect\hyperlink{ref-astor2014trait_dispersion}{2014})
\end{itemize}

\begin{quote}
\begin{itemize}
\tightlist
\item
  To test for deviations from random assembly, we used three null models
  and calcu- lated the standard effect size (SES; Gotelli and McCabe
  2002) as (observed a-Rao minus mean of expected a- Rao) divided by
  standard deviation of expected a-Rao. The observed and expected values
  were compared, and the significance was tested with one-sided
  permutation tests (with 999 randomizations) using the function
  ``as.randtest'' of the package ``ade4'' (Dray and Dufour 2007). In
  one-tailed null model tests, observed values of SES \textless{} 1.55
  (underdispersion) or \textgreater{}1.55 (overdispersion) indicate
  significant (a = 0.05) assembly pattern.
\item
  The randomization procedure was carried out with the trial swap method
  of Miklos and Podani (2004) implemented in R (R core team) with the
  \texttt{randomizeMatrix} function of the package \texttt{picante}
  (Kembel et al. 2010) with 999 randomizations.

  \begin{itemize}
  \tightlist
  \item
    \textbf{Model 1} : In the first null model, we randomized
    communities (species x plots matrix) by reshuffling the species
    identity among islands while keeping the same number of species per
    site and the same total species occurrence frequency in the whole
    region and calculated the abundance-weighted a-Rao for each random
    community. This represents the original Rao index comprising both
    functional richness and functional divergence.
  \item
    \textbf{Model 2} : For the second null model, we randomized the
    abundances among species within communities and calculated the
    abundance-weighted Rao. This converts the Rao into a pure divergence
    component.

    \begin{itemize}
    \tightlist
    \item
       Voir Botta-Dukát \& Czúcz
      (\protect\hyperlink{ref-botta_dukat2016}{2016}) pour pertinence de
      ce modèle 
    \end{itemize}
  \item
    \textbf{Model 3} : For the third null model, we again used the trial
    swap randomization, but calcu- lated the Rao based on species
    occurrences (presence/ absence) only. This resembles the functional
    richness component.
  \end{itemize}
\end{itemize}
\end{quote}

\begin{itemize}
\tightlist
\item
  Marteinsdóttir \emph{et al.}
  (\protect\hyperlink{ref-marteinsdottir2018}{2018})
\end{itemize}

\begin{quote}
For trait-divergence, we used randomization that swaps abundances across
all species occurring in the observed species pool

\begin{itemize}
\tightlist
\item
  Correspond à modèle C2 de Götzenberger \emph{et al.}
  (\protect\hyperlink{ref-gotzenberger2016randomizations}{2016}) et au
  modèle ``richness'' du package \texttt{Picante} qui est approprié pour
  détecter la divergence lié à l'effet d'une \emph{limiting similarity}
  si la randomization est faite au sein de chaque site.
\end{itemize}

For trait-convergence, we used a model that exchanges species abundances
across plots.

\begin{itemize}
\tightlist
\item
  Correspond au modèle C4 de Götzenberger \emph{et al.}
  (\protect\hyperlink{ref-gotzenberger2016randomizations}{2016}) et au
  modèle ``frequency'' du package \texttt{Picante} qui est approprié
  pour détecter la convergence lié à l'effet des filtres
  environnementaux si les gradients environnementaux sont suffisant et
  si le pool d'espèce randomisé permet d'inclure des espèces dans un
  habitat dans lequelle elle sont exclue en réalité du fait des
  processus de filtration
\end{itemize}
\end{quote}

\begin{center}\rule{0.5\linewidth}{\linethickness}\end{center}

\hypertarget{test}{\section{Test}\label{test}}

\subsection{randomizeMatrix}\label{randomizematrix}

\subsubsection{\texorpdfstring{Randomisation avec le modèle
``trialswap''}{Randomisation avec le modèle trialswap}}\label{randomisation-avec-le-modele-trialswap}

\begin{Shaded}
\begin{Highlighting}[]
\KeywordTok{data}\NormalTok{(mite)}

\NormalTok{## Diversité des sites}

\CommentTok{#specnumber(mite)}

\NormalTok{## Abondance par site}

\CommentTok{#rowSums(mite)}

\NormalTok{## Abondance par espèce}

\CommentTok{#colSums(mite)}

\NormalTok{## Occurrence par espèces}

\CommentTok{#specnumber(mite, MARGIN=2)}

\CommentTok{# Trialswap}
\CommentTok{#-----------}
\NormalTok{tmp <-}\StringTok{ }\KeywordTok{randomizeMatrix}\NormalTok{(mite, }\DataTypeTok{null.model=}\StringTok{"trialswap"}\NormalTok{, }\DataTypeTok{iterations =} \DecValTok{10000}\NormalTok{)}
\end{Highlighting}
\end{Shaded}

La matrice randomisée est-elle identique à l'originale?

\begin{Shaded}
\begin{Highlighting}[]
\CommentTok{# La matrice randomisée est-elle identique à l'originale?}
\KeywordTok{isTRUE}\NormalTok{(}\KeywordTok{all.equal}\NormalTok{(mite,tmp))}
\end{Highlighting}
\end{Shaded}

Diversité des sites préservée ?

\begin{Shaded}
\begin{Highlighting}[]
\CommentTok{# Diversité des sites préservée ?}
\KeywordTok{all}\NormalTok{(}\KeywordTok{specnumber}\NormalTok{(tmp) }\OperatorTok{==}\StringTok{ }\KeywordTok{specnumber}\NormalTok{(mite))}
\end{Highlighting}
\end{Shaded}

Abondance par site préservée ?

\begin{Shaded}
\begin{Highlighting}[]
\CommentTok{# Abondance par site préservée ?}
\KeywordTok{all}\NormalTok{(}\KeywordTok{rowSums}\NormalTok{(tmp) }\OperatorTok{==}\StringTok{ }\KeywordTok{rowSums}\NormalTok{(mite))}
\end{Highlighting}
\end{Shaded}

Abondance par espèce?

\begin{Shaded}
\begin{Highlighting}[]
\CommentTok{# Abondance par espèce?}
\KeywordTok{all}\NormalTok{(}\KeywordTok{colSums}\NormalTok{(tmp) }\OperatorTok{==}\StringTok{ }\KeywordTok{colSums}\NormalTok{(mite))}
\end{Highlighting}
\end{Shaded}

Occurence des espèces préservée ?

\begin{Shaded}
\begin{Highlighting}[]
\CommentTok{# Occurence des espèces préservée ?}
\KeywordTok{all}\NormalTok{(}\KeywordTok{specnumber}\NormalTok{(tmp, }\DataTypeTok{MARGIN=}\DecValTok{2}\NormalTok{)}\OperatorTok{==}\KeywordTok{specnumber}\NormalTok{(mite, }\DataTypeTok{MARGIN=}\DecValTok{2}\NormalTok{))}
\end{Highlighting}
\end{Shaded}

\subsubsection{\texorpdfstring{Randomisation avec le modèle
``richness''}{Randomisation avec le modèle richness}}\label{randomisation-avec-le-modele-richness}

\begin{Shaded}
\begin{Highlighting}[]
\CommentTok{# Richness}
\CommentTok{#---------}
\NormalTok{tmp <-}\StringTok{ }\KeywordTok{randomizeMatrix}\NormalTok{(mite, }\DataTypeTok{null.model=}\StringTok{"richness"}\NormalTok{, }\DataTypeTok{iterations =} \DecValTok{10000}\NormalTok{)}
\end{Highlighting}
\end{Shaded}

La matrice randomisée est-elle identique à l'originale?

\begin{Shaded}
\begin{Highlighting}[]
\CommentTok{# La matrice randomisée est-elle identique à l'originale?}
\KeywordTok{isTRUE}\NormalTok{(}\KeywordTok{all.equal}\NormalTok{(mite,tmp))}
\end{Highlighting}
\end{Shaded}

Diversité des sites préservée ?

\begin{Shaded}
\begin{Highlighting}[]
\CommentTok{# Diversité des sites préservée ?}
\KeywordTok{all}\NormalTok{(}\KeywordTok{specnumber}\NormalTok{(tmp) }\OperatorTok{==}\StringTok{ }\KeywordTok{specnumber}\NormalTok{(mite))}
\end{Highlighting}
\end{Shaded}

Abondance par site préservée ?

\begin{Shaded}
\begin{Highlighting}[]
\CommentTok{# Abondance par site préservée ?}
\KeywordTok{all}\NormalTok{(}\KeywordTok{rowSums}\NormalTok{(tmp) }\OperatorTok{==}\StringTok{ }\KeywordTok{rowSums}\NormalTok{(mite))}
\end{Highlighting}
\end{Shaded}

Abondance par espèce?

\begin{Shaded}
\begin{Highlighting}[]
\CommentTok{# Abondance par espèce?}
\KeywordTok{all}\NormalTok{(}\KeywordTok{colSums}\NormalTok{(tmp) }\OperatorTok{==}\StringTok{ }\KeywordTok{colSums}\NormalTok{(mite))}
\end{Highlighting}
\end{Shaded}

Occurence des espèces préservée ?

\begin{Shaded}
\begin{Highlighting}[]
\CommentTok{# Occurence des espèces préservée ?}
\KeywordTok{all}\NormalTok{(}\KeywordTok{specnumber}\NormalTok{(tmp, }\DataTypeTok{MARGIN=}\DecValTok{2}\NormalTok{)}\OperatorTok{==}\KeywordTok{specnumber}\NormalTok{(mite, }\DataTypeTok{MARGIN=}\DecValTok{2}\NormalTok{))}
\end{Highlighting}
\end{Shaded}

\subsubsection{\texorpdfstring{Randomisation avec le modèle
``frequency''}{Randomisation avec le modèle frequency}}\label{randomisation-avec-le-modele-frequency}

\begin{Shaded}
\begin{Highlighting}[]
\CommentTok{# Frequency}
\CommentTok{#---------}
\NormalTok{tmp <-}\StringTok{ }\KeywordTok{randomizeMatrix}\NormalTok{(mite, }\DataTypeTok{null.model=}\StringTok{"frequency"}\NormalTok{, }\DataTypeTok{iterations =} \DecValTok{10000}\NormalTok{)}
\end{Highlighting}
\end{Shaded}

La matrice randomisée est-elle identique à l'originale?

\begin{Shaded}
\begin{Highlighting}[]
\CommentTok{# La matrice randomisée est-elle identique à l'originale?}
\KeywordTok{isTRUE}\NormalTok{(}\KeywordTok{all.equal}\NormalTok{(mite,tmp))}
\end{Highlighting}
\end{Shaded}

Diversité des sites préservée ?

\begin{Shaded}
\begin{Highlighting}[]
\CommentTok{# Diversité des sites préservée ?}
\KeywordTok{all}\NormalTok{(}\KeywordTok{specnumber}\NormalTok{(tmp) }\OperatorTok{==}\StringTok{ }\KeywordTok{specnumber}\NormalTok{(mite))}
\end{Highlighting}
\end{Shaded}

Abondance par site préservée ?

\begin{Shaded}
\begin{Highlighting}[]
\CommentTok{# Abondance par site préservée ?}
\KeywordTok{all}\NormalTok{(}\KeywordTok{rowSums}\NormalTok{(tmp) }\OperatorTok{==}\StringTok{ }\KeywordTok{rowSums}\NormalTok{(mite))}
\end{Highlighting}
\end{Shaded}

Abondance par espèce?

\begin{Shaded}
\begin{Highlighting}[]
\CommentTok{# Abondance par espèce?}
\KeywordTok{all}\NormalTok{(}\KeywordTok{colSums}\NormalTok{(tmp) }\OperatorTok{==}\StringTok{ }\KeywordTok{colSums}\NormalTok{(mite))}
\end{Highlighting}
\end{Shaded}

Occurence des espèces préservée ?

\begin{Shaded}
\begin{Highlighting}[]
\CommentTok{# Occurence des espèces préservée ?}
\KeywordTok{all}\NormalTok{(}\KeywordTok{specnumber}\NormalTok{(tmp, }\DataTypeTok{MARGIN=}\DecValTok{2}\NormalTok{)}\OperatorTok{==}\KeywordTok{specnumber}\NormalTok{(mite, }\DataTypeTok{MARGIN=}\DecValTok{2}\NormalTok{))}
\end{Highlighting}
\end{Shaded}

\subsubsection{Bialn}\label{bialn}

\subsection{Permatswap / Commsimulator}\label{permatswap-commsimulator}

\begin{Shaded}
\begin{Highlighting}[]
\KeywordTok{data}\NormalTok{(mite)}

\NormalTok{tmp <-}\StringTok{ }\KeywordTok{permatswap}\NormalTok{(mite,}\DataTypeTok{method=}\StringTok{"abuswap"}\NormalTok{, }\DataTypeTok{fixedmar=}\StringTok{"columns"}\NormalTok{,}\DataTypeTok{times=}\DecValTok{1}\NormalTok{)}
\end{Highlighting}
\end{Shaded}

La matrice randomisée est-elle identique à l'originale?

\begin{Shaded}
\begin{Highlighting}[]
\CommentTok{# La matrice randomisée est-elle identique à l'originale?}
\KeywordTok{isTRUE}\NormalTok{(}\KeywordTok{all.equal}\NormalTok{(mite,tmp}\OperatorTok{$}\NormalTok{perm[[}\DecValTok{1}\NormalTok{]]))}
\end{Highlighting}
\end{Shaded}

Diversité des sites préservée ?

\begin{Shaded}
\begin{Highlighting}[]
\CommentTok{# Diversité des sites préservée ?}
\KeywordTok{all}\NormalTok{(}\KeywordTok{specnumber}\NormalTok{(tmp}\OperatorTok{$}\NormalTok{perm[[}\DecValTok{1}\NormalTok{]]) }\OperatorTok{==}\StringTok{ }\KeywordTok{specnumber}\NormalTok{(mite))}
\end{Highlighting}
\end{Shaded}

Abondance par site préservée ?

\begin{Shaded}
\begin{Highlighting}[]
\CommentTok{# Abondance par site préservée ?}
\KeywordTok{all}\NormalTok{(}\KeywordTok{rowSums}\NormalTok{(tmp}\OperatorTok{$}\NormalTok{perm[[}\DecValTok{1}\NormalTok{]]) }\OperatorTok{==}\StringTok{ }\KeywordTok{rowSums}\NormalTok{(mite))}
\end{Highlighting}
\end{Shaded}

Abondance par espèce?

\begin{Shaded}
\begin{Highlighting}[]
\CommentTok{# Abondance par espèce?}
\KeywordTok{all}\NormalTok{(}\KeywordTok{colSums}\NormalTok{(tmp}\OperatorTok{$}\NormalTok{perm[[}\DecValTok{1}\NormalTok{]]) }\OperatorTok{==}\StringTok{ }\KeywordTok{colSums}\NormalTok{(mite))}
\end{Highlighting}
\end{Shaded}

Occurence des espèces préservée ?

\begin{Shaded}
\begin{Highlighting}[]
\CommentTok{# Occurence des espèces préservée ?}
\KeywordTok{all}\NormalTok{(}\KeywordTok{specnumber}\NormalTok{(tmp}\OperatorTok{$}\NormalTok{perm[[}\DecValTok{1}\NormalTok{]], }\DataTypeTok{MARGIN=}\DecValTok{2}\NormalTok{)}\OperatorTok{==}\KeywordTok{specnumber}\NormalTok{(mite, }\DataTypeTok{MARGIN=}\DecValTok{2}\NormalTok{))}
\end{Highlighting}
\end{Shaded}

\begin{quote}
permatswap(x,method=``abuswap'', fixedmar=``columns'') équivaut à
randomizeMatrix(x,null.model==``trialswap'') Est-ce que ça donne la même
\end{quote}

\begin{Shaded}
\begin{Highlighting}[]
\KeywordTok{commsimulator}\NormalTok{(mite,}\DataTypeTok{method=}\StringTok{"r1"}\NormalTok{)}

\NormalTok{tmp <-}\StringTok{ }\KeywordTok{simulate}\NormalTok{(}\KeywordTok{nullmodel}\NormalTok{(mite,}\DataTypeTok{method=}\StringTok{"r1"}\NormalTok{), }\DataTypeTok{seed=}\DecValTok{1}\NormalTok{, }\DataTypeTok{nsim=}\DecValTok{10}\NormalTok{)}

\NormalTok{my_rao <-}\StringTok{ }\ControlFlowTok{function}\NormalTok{(x)\{}
\NormalTok{  tmp <-}\StringTok{ }\KeywordTok{rao.diversity}\NormalTok{(x)}
\NormalTok{  \}}

\KeywordTok{oecosimu}\NormalTok{(}\DataTypeTok{comm =}\NormalTok{ sipoo, }\DataTypeTok{nestfun =}\NormalTok{ , }\DataTypeTok{method =} \StringTok{"swap"}\NormalTok{,}
\DataTypeTok{burnin =} \DecValTok{100}\NormalTok{, }\DataTypeTok{thin =} \DecValTok{10}\NormalTok{, }\DataTypeTok{statistic =} \StringTok{"evals"}\NormalTok{)}
\end{Highlighting}
\end{Shaded}

\begin{center}\rule{0.5\linewidth}{\linethickness}\end{center}

\section*{Bilbiographie}\label{bilbiographie}
\addcontentsline{toc}{section}{Bilbiographie}

\hypertarget{refs}{}
\hypertarget{ref-astor2014trait_dispersion}{}
Astor, T., Strengbom, J., Berg, M.P., Lenoir, L., Marteinsdóttir, B. \&
Bengtsson, J. (2014). Underdispersion and overdispersion of traits in
terrestrial snail communities on islands. \emph{Ecology and evolution},
4, 2090--2102.

\hypertarget{ref-de_Bello_2011}{}
Bello, F. de. (2011). The quest for trait convergence and divergence in
community assembly: Are null-models the magic wand? \emph{Global Ecology
and Biogeography}, 21, 312--317.

\hypertarget{ref-botta_dukat2016}{}
Botta-Dukát, Z. \& Czúcz, B. (2016). Testing the ability of functional
diversity indices to detect trait convergence and divergence using
individual-based simulation. \emph{Methods in Ecology and Evolution}, 7,
114--126.

\hypertarget{ref-chalmandrier2013}{}
Chalmandrier, L., Münkemüller, T., Gallien, L., Bello, F., Mazel, F.,
Lavergne, S. \& Thuiller, W. (2013). A family of null models to
distinguish between environmental filtering and biotic interactions in
functional diversity patterns. \emph{Journal of Vegetation Science}, 24,
853--864.

\hypertarget{ref-dandrea_ostling2016}{}
D'andrea, R. \& Ostling, A. (2016). Challenges in linking trait patterns
to niche differentiation. \emph{Oikos}, 125, 1369--1385.

\hypertarget{ref-gotelli_mccabe2002}{}
Gotelli, N.J. \& McCabe, D.J. (2002). Species co-occurrence: A
meta-analysis of jm diamond's assembly rules model. \emph{Ecology}, 83,
2091--2096.

\hypertarget{ref-gotzenberger2016randomizations}{}
Götzenberger, L., Botta-Dukát, Z., Lepš, J., Pärtel, M., Zobel, M. \&
Bello, F. (2016). Which randomizations detect convergence and divergence
in trait-based community assembly? A test of commonly used null models.
\emph{Journal of Vegetation Science}, 27, 1275--1287.

\hypertarget{ref-grenie2017funrar}{}
Grenié, M., Denelle, P., Tucker, C.M., Munoz, F. \& Violle, C. (2017).
Funrar: An r package to characterize functional rarity. \emph{Diversity
and Distributions}, 23, 1365--1371.

\hypertarget{ref-kleyer2012methods}{}
Kleyer, M., Dray, S., Bello, F., Lepš, J., Pakeman, R.J., Strauss, B.,
Thuiller, W. \& Lavorel, S. (2012). Assessing species and community
functional responses to environmental gradients: Which multivariate
methods? \emph{Journal of Vegetation Science}, 23, 805--821.

\hypertarget{ref-laliberte2010FDis}{}
Laliberté, E. \& Legendre, P. (2010). A distance-based framework for
measuring functional diversity from multiple traits. \emph{Ecology}, 91,
299--305.

\hypertarget{ref-laliberte2013}{}
Laliberté, E., Norton, D.A. \& Scott, D. (2013). Contrasting effects of
productivity and disturbance on plant functional diversity at local and
metacommunity scales. \emph{Journal of Vegetation Science}, 24,
834--842.

\hypertarget{ref-marteinsdottir2018}{}
Marteinsdóttir, B., Svavarsdóttir, K. \& Thórhallsdóttir, T.E. (2018).
Multiple mechanisms of early plant community assembly with stochasticity
driving the process. \emph{Ecology}, 99, 91--102.

\hypertarget{ref-mason2013}{}
Mason, N.W., Bello, F., Mouillot, D., Pavoine, S. \& Dray, S. (2013). A
guide for using functional diversity indices to reveal changes in
assembly processes along ecological gradients. \emph{Journal of
Vegetation Science}, 24, 794--806.

\hypertarget{ref-mouillot2013indic}{}
Mouillot, D., Graham, N.A., Villéger, S., Mason, N.W. \& Bellwood, D.R.
(2013). A functional approach reveals community responses to
disturbances. \emph{Trends in ecology \& evolution}, 28, 167--177.

\hypertarget{ref-perronne2017}{}
Perronne, R., Munoz, F., Borgy, B., Reboud, X. \& Gaba, S. (2017). How
to design trait-based analyses of community assembly mechanisms:
Insights and guidelines from a literature review. \emph{Perspectives in
Plant Ecology, Evolution and Systematics}, 25, 29--44.

\hypertarget{ref-spasojevic2012}{}
Spasojevic, M.J. \& Suding, K.N. (2012). Inferring community assembly
mechanisms from functional diversity patterns: The importance of
multiple assembly processes. \emph{Journal of Ecology}, 100, 652--661.

\hypertarget{ref-violle2017functional_rarity}{}
Violle, C., Thuiller, W., Mouquet, N., Munoz, F., Kraft, N.J., Cadotte,
M.W., Livingstone, S.W. \& Mouillot, D. (2017). Functional rarity: The
ecology of outliers. \emph{Trends in ecology \& evolution}, 32,
356--367.

\hypertarget{ref-wesuls2012partialRLQ}{}
Wesuls, D., Oldeland, J. \& Dray, S. (2012). Disentangling plant trait
responses to livestock grazing from spatio-temporal variation: The
partial rlq approach. \emph{Journal of Vegetation Science}, 23, 98--113.

\hypertarget{ref-zhang2018}{}
Zhang, H., Chen, H.Y., Lian, J., John, R., Li, R., Liu, H., Ye, W.,
Berninger, F. \& Ye, Q. (2018). Using functional trait diversity
patterns to disentangle the scale-dependent ecological processes in a
subtropical forest. \emph{Functional Ecology}.

\hypertarget{ref-zhang2015}{}
Zhang, H., Qi, W., John, R., Wang, W., Song, F. \& Zhou, S. (2015).
Using functional trait diversity to evaluate the contribution of
multiple ecological processes to community assembly during succession.
\emph{Ecography}, 38, 1176--1186.


\end{document}
